
\newcommand{\jt}[2]{
\item[$\circ$]{#1}
  \begin{lstlisting}
    {#2}
  \end{lstlisting}
}

\subsection{nuSQUIDSAtm}

Atmospheric neutrino oscillations is a major and instrumental part of
research in contemporary neutrino physics. Experiments like
SuperKamiokande, IceCube, and Antares have used atmospheric neutrinos
to measure neutrino mass splittings and mixing angles. Furthermore,
proposed extensions like HyperKamiokande, PINGU, and ORCA ought to
improve the current measurements and have sensitivity to neutrino
neutrino mass ordering. This class allows to propagate a set of full
energy spectrum of neutrinos for a different number of zenith angles.
It implements functions to set easily the initial conditions and also
interpolations to get the fluxes.
\subsubsection{Constructors}

\begin{itemize}
\item[$\circ$] Constructor with {\ttf costh} range.
  \begin{lstlisting}
    template<typename... ArgTypes> nuSQUIDSAtm(double costh_min, 
                                               double costh_max,
                                               unsigned int costh_div, 
                                               ArgTypes&&... args)
  \end{lstlisting}
This constructor it creates a set {\ttf nuSQUIDS} or derived {\ttf
  nuSQUIDS} objects with a set of
{\ttf costh\_div} zenith angles from {\ttf costh\_min} to {\ttf
  costh\_max}, the last arguments are the argmunets of the base {\ttf
  nuSQUIDS} object.

\item[$\circ$] Constructor with {\ttf costh} array.
  \begin{lstlisting}
    template<typename... ArgTypes> nuSQUIDSAtm(marray<double,1> costh_array,
                                               ArgTypes&&... args):
  \end{lstlisting}
Similar to the last constructor but instead of define the boundaries
and the number of divisions the user should give as an argument an
array with the number of the zenith cosine.
\item[$\circ$] Constructing from a $\nu$SQuIDSAtm-HDF5 file
  \begin{lstlisting}
    nuSQUIDSAtm(std::string hdf5_filename);
  \end{lstlisting}
This constructor initializes {\ttfamily nuSQUIDS} from a 
previously generated $\nu$SQuIDS-HDF5 file. The result {\ttfamily nuSQUIDS} 
object will be given in {\it single} or {\it multiple} energy mode
depending on the HDF5 file configuration. {\ttfamily filepath} must specify the full
path of the HDF5 file. Furthermore,
{\ttfamily grp} specifies the location on the HDF5 file structure
where the object will be saved; by default
it will be saved on the {\ttfamily root} of the HDF5 file.
\item[$\circ$] Constructing from a $\nu$SQuIDS-HDF5 object
  \begin{lstlisting}
    nuSQUIDSAtm(nuSQUIDSAtm&&);
  \end{lstlisting}
Construct a {\ttfamily nuSQUIDSAtm} object from an existing {\ttfamily
  nuSQUIDSAtm} object through the {\ttfamily move} operator.

\end{itemize}


\subsubsection{Functions}
